\section{Introducao}

O volume de vendas de músicas em meio convencionais como o Rádio está perdendo seu papel como indicador de popularidade de artistas. Embora os números de vendas e execuções em rádios possam ter sido muito utilizados para medir a popularidade de um artista nas décadas de 50 e 60, essas variáveis tornaram-se cada vez mais pobres como indicadores de popularidade em detrimento ao rápido crescimento da divulgação da música em meio digital~\cite{Koenigstein:09}. Com o surgimento de novos meios, através das quais as comunidades estão expostas �  música, como as redes sociais, percebe-se a necessidade de reavaliar a forma como a popularidade de um artista é medida.
Segundo relatório da Federação Internacional da Indústria Fonográfica – IFPI, as receitas do mercado global de música gravada tiveram em 2015, crescimento de 3,2\% em relação ao ano anterior, atingindo US\$ 15,0 Bilhões. No mercado mundial, as vendas físicas caíram 4,5\% em 2015, já as receitas da área digital cresceram 10,2\%, e já representam mais da metade do faturamento com música gravada em 19 Países, incluindo o Brasil. O streaming é o formato com maior crescimento, representando 19\% do total das receitas fonográficas. O mercado de downloads representa 20\% do total das receitas fonográficas mundiais.	No Brasil, o mercado fonográfico  teve em 2015 aumento em suas receitas de 10,6\%, impulsionado pela continuidade do crescimento da área digital (+45,1\%). O decréscimo das vendas físicas (-19,3\%) e, em contrapartida, o desempenho significativo do mercado de música digital (+ 45,1\%) constatam que a distribuição de música gravada através de meios digitais está em franca ascensão, seja por streaming, downloads ou telefonia móvel. Segundo Paulo Rosa, Presidente da ABPD, O relatório do IFPI sobre o mercado fonográfico em 2015 confirma a tendência dos últimos anos que já apontava para um gradual amadurecimento do mercado de distribuição de música em meios digitais. Além disso, os números divulgados pela ABPD demonstram que o mercado brasileiro segue a mesma tendência do mercado mundial, com o setor digital sendo determinante para seu crescimento e já representando a maior parte de suas receitas.”
Assim, devido a nova configuração do mercado fonográfico, novas formas de se avaliar os artistas no cenário musical surgiram. Considerado uma revolução no \textit{marketing} da indústria fonográfica por~\cite{Malley:13}, o Next Big Sound é capaz de analisar o consumo de músicas de artistas através de uma enorme gama de pontos de venda e mídias digitais, incluindo Spotify, Pandora, Last.fm, Vevo, Facebook e Twitter. A empresa afirma que pode prever as vendas de álbuns com 20\% de precisão para 85\% dos artistas, atentando para o crescimento do artista no Facebook e outras medidas. O Next Big Sound, em parceria com a Billboard, provê seus dados para que a revista crie o \textit{ranking} Social 50, na qual apresenta os 50 artistas de cada semana mais populares nas mídias digitais Facebook, Pandora, Twitter, Last.fm e MySpace e Youtube \cite{Johnston:10} \cite{Nielsen:11}.
Outro fator que contribui para a popularidade de um artista é a análise de eventos os quais estão relacionados ao meio musical, como, por exemplo, o lançamento de uma música em um programa de televisão (TV) e a utilização dela em novelas e séries na TV. Ter espaço em uma novela, consequentemente, abre espaço na grade de programação da emissora, atingindo, por conseguinte, emissoras de rádio~\cite{Guerrini:10}.
Uma pesquisa divulgada pelo Next Big \footnote[11]{https://www.nextbigsound.com/industryreport/2013/}, referente ao ano de 2013, aponta que 91\% das bandas que existem no mundo ainda não foram descobertas pelo público. Isso ocorre, principalmente, porque as mídias digitais de maior acesso estão voltadas a artistas que estão no topo dos \textit{rankings}. Essa fatia de aproximadamente 1\% é que domina 87,3\% de todos os likes no Facebook e visualizações no Youtube e VEVO.
Com base no exposto, percebe-se a importância da mídia digital e da mídia de massa para se analisar a popularidade de artistas no mundo da música. Embora as pesquisas sobre as mídias digitais estejam em amplo crescimento, ainda há pouco material relacionado � s especificidades da produção e distribuição de conteúdo musical nessas mídias, como também � s práticas e estratégias de divulgação utilizadas por produtores e fãs em tais ambientes. Além disso, o potencial de medição da popularidade de um artista por meio das \textit{features} disponibilizadas por mídias digitais como \textit{likes} e \textit{dislikes} ainda permanece inexplorado no contexto da Recuperação de Informação \cite{chelaru:13}. A utilização desses dados, provenientes de diversas fontes, na geração de \textit{rankings}, pode gerar resultados diferentes daqueles que se têm convencionalmente.
Embora haja \textit{rankings} do meio fonográfico que utilizam das mídias digitais como o \textit{ranking} Social 50, observa-se que não há uma metodologia clara e transparente de como são calculados as posições dos \textit{rankings}. Além disso, observa-se que há sempre uma tendência desses \textit{rankings} considerarem apenas artistas que estão despontando, limitando a análise do mercado fonográfico e desconsiderando, por exemplo, o histórico da música ao longo dos anos e o que isso pode influenciar no mercado fonográfico atualmente. Foi pensando nisto que (referenciar minha pesquisa) desenvolveu uma metodologia de construção de rankings de artistas a partir da análise e mineração de dados coletados frequentemente de mídias digitais (Youtube, Last.fm, Letras, Twitter, Facebook, Vagalume, CifraClub e Rdio) e das mídias de massa TV, no intuito de avaliar a popularidade de um artista em análise. 
A pesquisa realizada por (referenciar minha pesquisa) desenvolveu a metodologia de construção de rankings  com base em uma lista de artistas internacionais. Visando analisar a metodologia de (referenciar minha pesquisa ) para o mercado fonográfico brasileiro, tendo em vista a importância do mesmo, este artigo busca realizar uma análise minuciosa do mercado fonográfico brasileiro utilizando a audiência do artista em várias mídias digitais e também na mídia TV.
Para isso, visa-se analisar os tipos de \textit{rankings} que podem ser desenvolvidos, estudando características e comportamentos dos artistas nas mídias digitais e TV, uma vez que artistas podem possuir perfis bem distintos em diferentes mídias, de forma a demandarem \textit{rankings} também distintos, ou podem possuir comportamentos semelhantes, necessitando-se desenvolver um \textit{ranking} que contenha artistas que possam ser comparáveis. 

\section{Trabalhos Relacionados}

Nesta seção são apresentados trabalhos no escopo de construção de \textit{rankings} de artistas, utilizando-se de mídias digitais, e também trabalhos que utilizam da relação entre as mídias digital e de massa para a construção de \textit{rankings}.
Estudiosos da música e sociólogos há muito tempo vêm demonstrado interesse nos padrões de consumo de música e sua relação com a situação socioeconômica por meio de \textit{features} em mídias digitais. Verboord (2016) salienta que as mídias digitais podem auxiliar na visibilidade de artistas, projetando o artistas tanto quanto a mídia televisiva. 
O trabalho de Yao Lu (2016) constrói um ranking de preferências de usuário com base em dados nas mídias digitais Last.fm e Douban.fm para auxílio na recomendação de músicas a usuários. Segundo os autores, seu algoritmo de recomendação é melhor se comparado a outros métodos da literatura que utilizam outras mídias digitais como o Deezer, por exemplo.
Os autores Zargerle etal. (2016), com base em um conjunto de dados reunidos a partir do Twitter e os gráficos da Billboard ao longo de 2014 e 2015, analisam se há a relação entre o Twitter e os gráficos Billboard no que diz respeito a se tweets poderiam ser utilizados para prever futuros gráficos da Billboard. Os resultados experimentais mostraram que, em princípio, as séries temporais Twitter e Billboard para músicas compartilham uma correlação moderada. Quanto ao poder preditivo dos métodos, incorporar dados do Twitter aos dados da Billboard reduz significativamente o erro de previsão. 
Yekyung Kim, Bongwon Suh, and Kyogu Lee (2014) também abordam a relação entre os comportamentos de escuta de música de usuários do Twitter e os rankings da Billboard. Os resultados da pesquisa mostram que o número de tweets diários sobre uma canção específica e artista pode ser efetivamente usado para prever classificações Billboard e hits. Esta pesquisa sugere que o comportamento de escuta de música dos usuários no Twitter está altamente correlacionado com tendências de música geral e poderia desempenhar um papel importante na compreensão dos padrões de consumo de música dos consumidores. Além disso, acreditam que o comportamento de escuta de música dos usuários do Twitter pode ser aplicado no campo de Music Information Retrieval (MIR).
O trabalho de \cite{Grace:07} propôs uma abordagem que utiliza técnicas de mineração de dados em texto para medir a popularidade e construir \textit{rankings} de artistas a partir da análise dos comentários dos ouvintes na rede social MySpace, uma mídia musical popular. Eles demonstraram ter obtido resultados mais próximos � queles que seriam gerados manualmente por usuários (estudantes universitários), do que os que são gerados a partir dos métodos utilizados pela revista Billboard.
O trabalho de \cite{Jordanous:14} aborda a criação de um \textit{ranking} de artistas do gênero Música Eletrônica utilizando-se de dados do SoundCloud, uma mídia digital utilizada para, entre outras coisas, comentar sobre o trabalho de artistas. Assim, os autores utilizam os comentários, \textit{likes} e compartilhamentos para ordenar os artistas em um \textit{ranking} dos artistas mais comentados, utilizando-se de técnicas como o PageRank. Acredita-se que construindo metodologias desse tipo, consegue-se abstrair \textit{insights} sobre a cultura social do público-alvo da análise.
Os autores do trabalho \cite{Bryan:11} construíram um \textit{ranking} de artistas com base no \textit{website} WhoSampled\footnote[13]{http://www.whosampled.com}, um \textit{website} de informações sobre o meio musical. Para isso, utilizaram várias características de artistas e gêneros musicais e adotaram a métrica \textit{PageRank} para permitir a interpretação e descrição de padrões de influência musical, tendências e características de músicas.
O Next Big Sound\footnote[15]{http//www.nextbigsound.com}, conforme já explanado sobre suas características, construiu o \textit{ranking} ``Social 50'', o qual leva em consideração a interação dos usuários com um artista nas mídias digitais. Além desse serviço, a MTV, de acordo com \cite{Kaufman:10}, também provê \textit{rankings} que levam em consideração essa análise, além da vendagem de CDs e audiência em rádios. 
Gravadoras influentes como a Sony estão assinando serviços como aqueles disponibilizados pelo Next Big Sound. Embora a quantidade de tempo que as pessoas passam ouvindo música tem aumentado devido ao grande número de canais disponíveis, muitas músicas podem chegar ao topo das paradas da Billboard com um número modesto de vendas, uma vez que em torno de um quarto das pessoas não pagam pela música que ouvem, devido aos serviços gratuitos de música por exemplo. Assim, essas gravadoras estão aderindo �  mudança de ``ouvir'' os ouvintes para entender como as pessoas querem consumir música.
Por fim, \cite{moss:14} apresentam um serviço de recomendação de música em que, para prover músicas ao usuário, utiliza-se da convergência entre mídias. Há dois \textit{rankings} interessantes a serem citados. O primeiro leva em consideração a popularidade do artista e o segundo considera as músicas de um artista. Ambos os \textit{rankings} consideram: informações advindas do rádio, seja ele tradicional ou via \textit{web}; a relação de um vídeo de determinado artista que está presente em mídias digitais como o Youtube e, ao mesmo tempo, na mídia televisiva em emissoras como a MTV; e informações fornecidas manualmente por especialistas para a categorização de artistas.
O trabalho de (citar minha pesquisa), no qual este trabalho está fortemente embasado, propõe uma metodologia de construção de rankings de artistas utilizando-se de dados de mídias digitais (Facebook, Youtube, Twitter, Letras, Last.fm) e da mídia TV (realizando uma análise da participação de artistas em vários canais de TV como Globo e SBT. Entre os rankings construídos, pode-se citar desde rankings que levam em consideração gêneros musicais específicos e tempos de carreira, desde métricas construídas a partir de métodos de predição. Diferentemente de trabalhos como os de Zargerle et al. (2016) e Yekyung Kim, Bongwon Suh, and Kyogu Lee (2014), o intuito deste trabalho não foi o de comparar os rankings produzidos pela metodologia desenvolvida com os rankings de empresas como a Billboard, uma vez que se acredita que não um ranking que exprime uma verdade absolta. O intuito foi a construção de uma metodologia que demonstrasse várias possibilidades de classificação de artistas que explorassem diferentes possibilidades a partir de dados de mídias digitais e de massa como a TV, demonstrando todos os passos necessários para a análise de um artista. Os resultados do artigo mostram que diferentes rankings podem ser construídos utilizando-se dados de mídias digitais e de massa, e que exprimem a realidade tanto quanto rankings respeitados como a Billboard, apesar de mostrar resultados diferentes.


\section{Metodologia}
O presente trabalho, assim como os demais, visa desenvolver uma metodologia de construção de rankings de artistas a partir da análise de dados de mídias digitais e de TV. No entanto, é utilizado dados de artistas do mercado fonográfico brasileiro, tendo em vista sua importância econômico-social. Dessa forma, visa-se analisar como a metodologia desenvolvida por (citar meu trabalho) se adequa para o mercado brasileiro, bem como analisar novas métricas aplicadas a este mercado.
Esta seção apresenta a metodologia de construção de \textit{rankings} agregados de artistas do meio fonográfico. A construção desses \textit{rankings} é realizada utilizando-se dados de mídias digitais, como \textit{likes} e exibições, levando em consideração o perfil dos artistas nas mídias digitais e na mídia televisiva, além de informações de conteúdo como gênero, nacionalidade, anos de carreira e a influência da mídia televisiva. Os diferentes tipos de \textit{rankings} propostos serão apresentados na Subseção “Rankings Desenvolvidos”. Vale ressaltar que a metodologia é escalável, podendo-se utilizar outras mídias digitais diferentes daquelas utilizadas neste trabalho bem como dados advindos de rádio, além de uma quantidade maior ou menor de artistas e diferentes características como anos de carreira e estilos musicais. Basta-se para isso aplicar a metodologia desenvolvida no trabalho para verificar a viabilidade de acrescentar novos dados. Vale salientar que para este trabalho, dos 1000 artistas na base de dados, foram considerados apenas os brasileiros (286 artistas).
A Figura~\ref{fig:rankingAgregado} apresenta a metodologia de construção de \textit{rankings} de artistas, que é dividida em vários passos. De maneira sucinta, a metodologia consiste em selecionar um conjunto de artistas a partir de uma lista. Posteriormente, coleta-se informações desses artistas na Wikipedia, como gênero e nacionalidade, e por meio de monitoramento no Facebook e canal do Youtube oficiais dos artistas para atualizar dados de suas carreiras. Após, passa-se para a coleta da audiência dos artistas nas mídias digitais (dados como likes, dislikes, visualizações). Em seguida, gera-se os rankings dos artistas em cada mídia digital. Os dados provenientes das mídias digitais são utilizados para a construção dos rankings individuais através do cálculo de ranking a partir da métrica Visibilidade'' \cite{Diego:12} (métrica cujo cálculo se dá com o somatório das \textit{features} das mídias digitais, como quantidade de \textit{likes}, visualizações e seguidores). Nessa etapa, também se adiciona dados de artistas monitorados na programação da TV brasileira. Por fim, como último passo tem-se a geração dos \textit{rankings} combinados'', o qual tem a função de agregar os dados vindos do passo anterior, relativos � s mídias digitais e da TV. Essa etapa recebe os \textit{rankings} de artistas de cada mídia digital gerados no passo anterior e, por meio de cálculos estatísticos (utilização do Coeficiente de Correlação e \textit{Rankings} de Spearman~\cite{Cohen:88}), os dados são agregados.  Assim, é gerado \textit{rankings} agregados de artistas.

O algoritmo que computou a agregação é apresentado no Algoritmo 1, onde $RM$ é uma matriz contendo os \textit{rankings} de cada artista em cada uma das oito mídias digitais (obtidos no passo 5 anterior) e $a \in A$ é o conjunto de artistas. O algoritmo recebe como entrada a Matriz $RM$ que contém o conjunto de artistas $A$ com suas respectivas posições em cada mídia digital. É computada a mídia das posições nos \textit{rankings} das oito mídias digitais consideradas para cada artista e são armazenadas no vetor $rank$. Posteriormente, esses valores são ordenados em ordem crescente, sendo os valores das mídias das posições normalizados com valores de 1 até a quantidade de artistas na base.  O \textit{ranking} agregado é retornado. O que difere em cada um é como se realizaram as médias das posições de cada artista nos \textit{rankings} individuais em cada mídia digital para agregação desses \textit{rankings} representados por $p(a)$. As médias são realizadas de quatro maneiras diferentes: média simples das posições dos artistas nos \textit{rankings} individuais das mídias digitais; média ponderada pelo Coeficiente de Correlação de \textit{Ranking} de Spearman entre as mídias; média ponderada pelo Coeficiente de Correlação de \textit{Ranking} de Spearman entre os \textit{rankings} de dias subsequentes no histórico de dias coletados em cada mídia; média ponderada pela posição da mídia digital no \textit{ranking} do website Alexa.

\section{Rankings Desenvolvidos}

A Tabela~\ref{tab49} desta subseção apresenta os parâmetros considerados nos \textit{rankings} propostos. Cada coluna da tabela consiste em: \textit{Ranking}, Montante, Período/Idade, Nº de Artistas na base. A coluna \textit{Ranking} consiste no tipo de \textit{ranking} construído, a coluna Montante leva em consideração a utilização ou não dos montantes de dados das mídias digitais (o valor total dos dados de cada mídia digital para cada artista), a coluna Período/Idade consiste no período considerado para análise (quantos anos de carreira do artista ou a quantidade de tempo considerada para a construção dos \textit{rankings}) e Nº de Artistas na Base consiste na quantidade de artistas que possuem as características de cada \textit{ranking} considerado.
 
Para este trabalho, um ranking será melhor detalhado, uma vez que este foi construído especificamente para o mercado fonográfico brasileiro: o Ranking de Exposição das Mídias Digitais e TV. O  \textit{Ranking} de Exposição da Mídias Digitais e TV foi construído levando em consideração os artistas que foram em algum programa de televisão dos canais abertos no ano de 2014 (Tabela X). Dos 1000 artistas, foram encontrados 162 com essa característica. As características obtidas no passo de “Descoberta de Novos Sucessos” da metodologia proposta auxiliam na descoberta dos artistas que estão em evidência na mídia, contribuindo para a contabilização dos artistas que foram em algum programa de TV. 

 

A análise feita a partir da participação de um artista em um programa de TV foi realizada a partir do monitoramento dos programas veiculados aos finais de semana, englobando programas com audiência variada e diferentes públicos e idades, das cinco principais emissoras, conforme visto a seguir. Esse monitoramento é realizado através da pesquisa manual em \textit{websites} ou com o auxílio do sistema do passo ``Descoberta de Novos Sucessos'', dos artistas que vão participar dos programas de finais de semana dos canais em monitoramento. Ao encontrar um artista que irá participar do programa de TV, ele é contabilizado na relação dos artistas a serem monitorados para a construção do \textit{Ranking} de Exposição de Mídia. 
A Equação \ref{eq:rMidia} apresenta o cálculo da posição do artista levando em consideração a Mídia TV. O cálculo é realizado através do somatório da quantidade de vezes que o artista foi nos programas de TV multiplicado pela audiência média desses programas no ano considerado.

 

onde $pos$ é a posição do artista no \textit{Ranking} de Exposição de Mídia, $p$ é o número de programas que o artista participou, $q$ é a quantidade de vezes que um artista apareceu em um determinado programa de TV e $\bar{h}$ é a audiência média de cada programa onde o artista participou.
O \textit{Ranking} de Exposição das Mídias Digitais e TV leva em consideração os dados obtidos do monitoramento do artista na TV, que nesse caso é utilizado como peso para a construção dos \textit{rankings} individuais (Equação \ref{eq:rMidia}). O \textit{ranking} também leva em consideração o montante dos dados dos artistas nas mídias digitais, no qual foi considerando o ano de 2014, por se ter acesso �  aparição dos artistas na TV a partir desse ano. Assim, o montante é multiplicado pelo peso.

\section{Experimentos}

Nesta seção, serão apresentados alguns dos principais rankings gerados para este estudo de caso devido ao limitado de espaço, levando em consideração a caracterização dos artistas descrita anteriormente. Para o estudo de caso, sero demonstrados 30 artistas em cada ranking.

\subsection{Ranking da Perenidade}

Esta subseção apresenta o \textit{Ranking} da Perenidade, que considera artistas surgidos há mais de 20 anos. A Tabela X apresenta os artistas presentes no \textit{ranking} da Perenidade e traz no topo do \textit{ranking} artistas consagrados no cenário musical nacional (como Roberto Carlos, Gilberto Gil, Chico Buarque, Caetano Veloso e Legião Urbana entre os 30 artistas melhor classificados). Percebe-se que o ranking da perenidade mostra uma realidade condizente com o mercado fonográfico nacional tanto quanto o mercado internacional, conforme mostrado no trabalho de (citar meu artigo). É interessante observar que a revista Rolling Stone aponta todos esses artistas citados como sendo os maiores artistas brasileiros de todos os tempos em ranking denominado “100 maiores artistas da música brasileira”. Percebe-se que dois terços dos artistas no ranking são do gênero Rock e MPB, gêneros que historicamente foram marcantes no Brasil.

\subsection{Ranking da Revelação}

Esta subseção apresenta o\textit{ Ranking} de Revelação, onde os artistas que estão em evidência no momento da análise (no caso, dezembro de 2014). A Tabela X apresenta os artistas presentes no \textit{ranking} de Revelação, onde dois terços dos artistas são de gêneros como Sertanejo, Funk e Arrocha, gêneros em ascensão no mercado brasileiro (achar referência). É interessante observar que vários artistas que fazem sucesso atualmente estão presentes neste ranking, como é o caso da dupla Sertaneja Simone e Simaria e cantora de Funk Ludmila. Outro fato interessante é a aparição da banda Malta, que foi a banda revelação a ganhar o programa SuperStar da rede Globo.

\subsection{Ranking de Tendência}

Esta subseção apresenta o \textit{Ranking} de Tendência, que considera a predição de posições de \textit{rankings} por meio de técnicas de regressão, utilizando-se série de dados históricos (dias da semana contendo as posições dos artistas no \textit{ranking}). A Tabela X apresenta o \textit{Ranking} de Tendência. Quase metade dos 30 artistas do topo do ranking (14) são do gênero Funk, Arrocha e Sertanejo, demonstrando a difusão desses gêneros na cultura brasileira, como mostrado também no ranking de Revelação. É interessante observar que artistas apontados como possíveis \hits" do carnaval de 2014 pelo Youtube, como o cantor Pablo, ou pela emissora Globo, como as cantoras Ludmilla e Valesca Popozuda, apresentaram-se entre os 30 melhores classificados no ranking do final do mês de dezembro, a dois meses do início do carnaval.

\subsection{Ranking da Mídia Digital e TV}

Há também o ranking considerando os artistas que apareceram em algum programa de TV, levando em consideração a quantidade de vezes que um artista foi em algum programa, a audiência e os dados das mídias digitais analisadas do mesmo, o Ranking da Exposição das Mídias Digitais e TV. A Tabela X apresenta artistas do gêneros populares (classificação dada por um estudo de categorização dos artistas que se leva em consideração  a popularidade e classe econômicas dos consumidores de músicas- referenciar) diversos, como Sertanejo e Funk. Artistas que apresentam músicas em novela e aparecem sempre na mídia ocupam o topo do ranking, caso das cantoras Anitta, Ludmila e Malta. Essa última banda ganhou notoriedade participando de um programa de TV na rede mais famosa do Brasil, a Globo, o SuperStar conforme já dito, além de ter tido sua música de trabalho como tema de abertura de uma novela na emissora.

\subsection{Ranking de Emergentes}

Esta subseção apresenta o Ranking de artistas Emergentes, que mostra o ranking dos artistas Emergentes em franca ascensão, que já despontaram no cenário musical e crescem em uma taxa maior que os artistas perenes ou modernos. Neste ranking é interessante notar que 16 artistas são do gênero Gospel. Nos últimos anos observa-se a ascensão dos artistas da música Gospel, o que pode ser notado pelo gráfico. 

\subsection{Relação entre os Rankings Construídos e Rankings do Meio Fonográfico}

Esta subseção apresenta o estudo comparativo dos rankings de artistas construídos neste
trabalho com os rankings de artistas construídos pela Billboard Brasil, importante veículo de comunicação que aufere a popularidade de um artista. Este estudo não tem o intuito de mostrar que os rankings gerados neste trabalho são melhores que os rankings fornecidos pela Billboard, mas apenas comparar suas características. Os rankings construídos neste trabalho, que são comparados com os rankings fornecidos por empresas externas, são os rankings do dia 25 de dezembro de 2014. São considerados os 10 artistas melhores classificados de cada ranking.
Os rankings a serem comparados são o ranking da Billboard Hot 100 – Brasil com o ranking de Tendência. Este ranking da Billboard leva em consideração, além de execuções dos artistas em rádios, também as vendas de CD e dados de mídias digitais. Ambos os rankings medem a audiência semanal do artista, indicando artistas que estão em voga. O ranking da Billboard leva em consideração 100 artistas, mas para fins de comparação foram considerados os 30 primeiros artistas. Observa-se nos dois rankings que existem 16 artistas em comum considerando os 30 artistas do topo. Em ambos predominam-se gêneros Populares como o Sertanejo e o Pagode, dando destaque para o primeiro (23 artistas no ranking da Billboard e 18 no ranking de Tendência).
Não se pode dizer que os rankings deste trabalho condizem mais com a realidade do que aqueles já existentes, porém, ao analisar que dentre os próprios rankings já consagrados existem vários, como os construídos pela Billboard, observa-se que não há apenas uma forma de construir rankings universalmente aceita. 
Ter uma metodologia aberta onde os estudiosos do meio fonográfico possam analisar diversos parâmetros, mídias e artistas, mostra que a metodologia desenvolvida neste trabalho um importante insumo para a tomada de decisão em um mercado cada vez mais dinâmico e exigente como mercado fonográfico brasileiro.

\section{Conclusions}
This paragraph will end the body of this sample document.
Remember that you might still have Acknowledgments or
Appendices; brief samples of these
follow.  There is still the Bibliography to deal with; and
we will make a disclaimer about that here: with the exception
of the reference to the \LaTeX\ book, the citations in
this paper are to articles which have nothing to
do with the present subject and are used as
examples only.
%\end{document}  % This is where a 'short' article might terminate



\appendix
%Appendix A
\section{Headings in Appendices}
The rules about hierarchical headings discussed above for
the body of the article are different in the appendices.
In the \textbf{appendix} environment, the command
\textbf{section} is used to
indicate the start of each Appendix, with alphabetic order
designation (i.e., the first is A, the second B, etc.) and
a title (if you include one).  So, if you need
hierarchical structure
\textit{within} an Appendix, start with \textbf{subsection} as the
highest level. Here is an outline of the body of this
document in Appendix-appropriate form:
%\subsection{Introduction}
%\subsection{The Body of the Paper}
%\subsubsection{Type Changes and  Special Characters}
%\subsubsection{Math Equations}
%\paragraph{Inline (In-text) Equations}
%\paragraph{Display Equations}
%\subsubsection{Citations}
%\subsubsection{Tables}
%\subsubsection{Figures}
%\subsubsection{Theorem-like Constructs}
%\subsubsection*{A Caveat for the \TeX\ Expert}
%\subsection{Conclusions}
%\subsection{References}
Generated by bibtex from your \texttt{.bib} file.  Run latex,
then bibtex, then latex twice (to resolve references)
to create the \texttt{.bbl} file.  Insert that \texttt{.bbl}
file into the \texttt{.tex} source file and comment out
the command \texttt{{\char'134}thebibliography}.
% This next section command marks the start of
% Appendix B, and does not continue the present hierarchy
\section{More Help for the Hardy}

Of course, reading the source code is always useful.  The file
\path{acmart.pdf} contains both the user guide and the commented
code.

\begin{acks}
  The authors would like to thank Dr. Yuhua Li for providing the
  matlab code of  the \textit{BEPS} method. 

  The authors would also like to thank the anonymous referees for
  their valuable comments and helpful suggestions. The work is
  supported by the \grantsponsor{GS501100001809}{National Natural
    Science Foundation of
    China}{http://dx.doi.org/10.13039/501100001809} under Grant
  No.:~\grantnum{GS501100001809}{61273304}
  and~\grantnum[http://www.nnsf.cn/youngscientsts]{GS501100001809}{Young
    Scientsts' Support Program}.

\end{acks}
